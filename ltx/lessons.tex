David:
\begin{itemize}
\item It is important to be familiar with the language we're working with, especially the debugging tools. The \texttt{OCAMLRUNPARAM} environment variable and running \texttt{ocamlyacc} with \texttt{-v} were invaluable tools at our disposal. 
\item We should code to the LRM. This means having the LRM finalized before coding begins. During a lot of situations when we were implementing the language, we realized something in the LRM was incomplete, incompatible, or just inaccurate. It is important to realize these issues early on. In general, having a complete and well thought-out spec is one of the more important parts of the software development cycle. 
\end{itemize}

Jonathan:
\begin{itemize}
\item Everyone needs to be able to work independently and be productive. But joint coding sessions have a lot of benefits too. Not only does it force everyone to set aside some time to work on the project for a few hours, but everyone can provide each other with immediate feedback and help with debugging, figuring out how to implement something etc. These factors tend to make such group sessions very productive. 
\item Unit tests are amazing. There's a good reason software engineers use them. Particularly in the middle of the development process, writing even 2 or 3 non-trivial tests is pretty much guaranteed to find at least one bug. 
\end{itemize}

Hans:
\begin{itemize}
\item Possibly the only reason we have a complete and working compiler--turned in on time--is that we recognized how lofty our initial dreams were and chose to limit our project's scope. Through careful planning we kept our grammar small and left frivolous niceties, such as strings, unimplemented unless we somehow finished early. Our small language is now complete and well-tested.
\item Code ownership can lead to ``blocking''--preventing someone else from completing their part, because they depend on your finishing. Using Git and communicating early and often with your team will make collaboration much easier--across modules and files. Pair programming is sometimes the perfect productivity boost you need to eliminate blocking.
\end{itemize}

Harley:
\begin{itemize}
\item The LRM is always authoritative and reliable--- {\it except when it's not.} Then, it needs to be corrected. Testing is largely validating that your language follows the LRM, but part of testing is  also verifying the completeness  and consistency of the LRM. 
\item Don't forget to test for parsing and compilation errors you {\bf want} to happen. If you only test the output of well-formed programs, you won't find out that your compiler happily compiles invalid code into a meaningless blob of executable gubbins.
\end{itemize}

\subsection{Advice for the Future}
\begin{itemize}
\item {\bf On getting a head start}\\
It is, of course, easy for everyone to say ``start early.'' It's similarly easy to think ``ok, we're going to start early.''  It's definitely harder to actually get started as early as you intend. Some ideas:
\begin{itemize}
\item Create some deadlines early on for your team for meeting and creating some initial tests, and skeletal implementations of your compiler. 
\item Send email to the  team about the project when the discussion goes quiet. It can be about anything: questions you have, or when the next meeting is occurring. Don't wait for your leader or others to send an email; often your activity will motivate or oblige your team members to become active as well.
\end{itemize}
\item {\bf Choose something you care about}\\
Or at least choose something that you find interesting. It will make the whole experience much more fun!
\end{itemize}