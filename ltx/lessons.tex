David. Todo.\\

Jonathan. Todo.\\

Hans. Todo.\\

Harley:
\begin{itemize}
\item The LRM is always authoritative and reliable--- {\it except when it's not.} Then, it needs to be corrected. Testing is largely validating that your language follows the LRM, but part of testing is  also verifying the completeness  and consistency of the LRM. 
\item Don't forget to test for parsing and compilation errors you {\bf want} to happen. If you only test the output of well-formed programs, you won't find out that your compiler happily compiles invalid code into a meaningless blob of executable gubbins.
\end{itemize}

\subsection{Advice for the Future}
\begin{itemize}
\item {\bf On Getting a Head Start}\\
It is, of course, easy for everyone to say "start early." It's similarly easy to think "ok, we're going to start early."  It's definitely harder to actually get started as early as you intend. Some ideas:
\begin{itemize}
\item Create some deadlines early on for your team for meeting and creating some initial tests, and skeletal implementations of your compiler. 
\item Send email to the  team about the project when the discussion goes quiet. It can be about anything: questions you have, or when the next meeting is occuring. Don't wait for your leader or others to send an email; often your activity will motivate or oblige your team members to become active as well.
\end{itemize}
\end{itemize}