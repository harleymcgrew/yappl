Probabilistic programming languages (PPLs) have grown increasingly popular in the machine learning research in recent years because they allow for the concise definition of complex statistical models. PPLs provide mechanisms for defining generative Bayesian models and conditionally sampling from them. The two key features the PPLs have to accomplish this are conditional evaluation and memoization. Conditional evaluation allows a function to be evaluated conditional on some predicate of its return value being true (this corresponds to conditional sampling from the model). A memoized function remembers what value it returned for previously evaluated argument values and always returns the same value in the future given those arguments. Memoization is useful because it allows elements from ``infinite'' random structures (like lists or matrices) to be generated lazily (as needed). 

YAPPL (Yet Another Probabilistic Programming Language) is inspired primarily inspired by the probabilistic programming language Church, an implementation of a pure subset of Scheme (a dialect of Lisp) for generating models using probabilistic functions. Church relies on the standard Lisp syntax, however, which can be unintuitive and difficult to read. The syntax of YAPPL is inspired by OCaml and contains special constructs for the probabilistic elements of the language, which makes it more approachable and human-readable than Church. In particular, conditional evaluation and memoization are directly built into the YAPPL language. This makes it easier to write and understand the probabilistic models written in YAPPL. 

YAPPL is an almost pure functional programming language. The only functions that produce side-effects are those explicitly built into the language: printing, seeding the random number generator, generating random numbers, and functions defined via memoization. 

We hope YAPPL will serve to demonstrate how writing a PPL from the the ground up can produce that is accessible and useful to a wide audience. 
 

