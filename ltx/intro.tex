Probabilistic programming languages have grown increasingly popular in recent years 
because they allow for the concise definition of complex statistical models. They also 
provide tools for sampling the (usually Bayesian) models. YAPPL is inspired by the 
probabilistic programming language Church, an implementation of a pure subset of Scheme 
(a dialect of Lisp) for generating models using probabilistic functions. Church relies 
on the standard Lisp syntax, which is unintuitive and difficult to read. The syntax of 
YAPPL is inspired by OCaml and contains special constructs for the probabilistic 
elements of the language, which makes it more approachable and human-readable than Church. 

HANSEI is a domain specific language library for ML that does implement some probabilistic 
functionality. YAPPL differentiates itself from HANSEI by providing clean, native syntax 
for the representation of stochastic functions, memoization, and conditional sampling. 