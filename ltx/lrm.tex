This is our language reference manual. 

\subsection{Lexical conventions}

As our syntax is inspired by OCaml, many of our lexical conventions follow those of that language. YAPPL has four kinds of tokens: identifiers, keywords, constants, and expression operators. Whitespace such as blanks, tabs, and newlines are ignored and serve to separate tokens. Comments are also ignored.

\subsubsection{Comments}

A single \texttt{\#} indicates that all succeeding characters shall be considered part of a comment and ignored until a newline is encountered. \\
\\
Immediately following a newline, a series of three \texttt{\#\#\#} indicates that all succeeding characters shall be considered part of a comment until another series of three \texttt{\#\#\#} is encountered. Note that newlines are ignored following the \texttt{\#\#\#}, which essentially delimits multi-line comments.

\subsubsection{Identifiers}

An identifier is a series of alphabetical letters and digits; the first character must be alphabetic. 

\subsubsection{Keywords}

The following identifiers are reserved as keywords and may not be used otherwise:
\begin{table}[htdp]
\begin{tabular}{c c c}
\texttt{fun} & \texttt{if} &\texttt{match} \\
\texttt{int} & \texttt{then} & \texttt{with} \\
\texttt{bool} & \texttt{else} &\texttt{case} \\
\texttt{float} & \texttt{in} \\
\texttt{true} & \texttt{false} \\
\end{tabular}
\label{default}
\end{table}%



\subsubsection{Constants}

YAPPL only supports numeric constants (integers and floats).

\subsection{Types}

(Hans)

\subsection{Operations}

\subsubsection{Value definitions (\texttt{=})}
\texttt{float:q = .9;}\\
\\
This assigns a value to a variable. 

\subsubsection{Sampling (\texttt{\textasciitilde})}
\texttt{\textasciitilde bin;}\\
\\
This samples \texttt{bin} and returns the value.

\subsubsection{Binding (\texttt{= \textasciitilde })}
\texttt{int:x = \textasciitilde geom q;}\\
\\
This binds a variable to a function's return value. In this case, \texttt{x} is bound to the return value of the function \texttt{geom} evaluated with parameter \texttt{q}.

\subsubsection{Function definition (\texttt{=})}
\texttt{fun int:oneOrTwo float:q = geom q;}\\
\\
This defines a function \texttt{oneOrTwo} that samples from \texttt{geom q}. It has a return type of \texttt{int} and takes a single parameter \texttt{q} of type \texttt{float}. 

\subsubsection{Conditioning (\texttt{|})}
\texttt{fun int:oneOrTwo float:q = geom q | @ = 1 or @ = 2;}\\
\\
This is identical to the previous \texttt{oneOrTwo} except we now condition that the sample be 1 or 2.

\subsubsection{Memoized function definition (\texttt{:=})}
\texttt{fun int:f int:n := ~geom .9;}\\
\\
\texttt{f} is a memoized function that returns the same value for each sample.

\subsection{Expressions}

(Harley)

\subsection{Functions}

(Jonathan)