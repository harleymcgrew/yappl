
\subsection{Notation}

Through the document, \nterm{nonterminals} are in brown italics and \term{terminals} are in light blue monospace. Regular expression-like constructs are used to simplify grammar presentation and are in black. Brackets \texttt{[]} are used to indicate optional parts of productions, curly braces \texttt{\{\}} indicate portions of productions that can appear zero or more times, and parentheses \texttt{()} indicate grouping, with a vertical bar \texttt{|} separating options. 

\subsection{Definition of a program}

A program is a sequence of one or more expressions.

\subsection{Lexical conventions}

As syntax of YAPPL is inspired by OCaml, many of the lexical conventions follow those of that language. YAPPL has four kinds of tokens: identifiers, keywords, constants, and expression operators. Whitespace such as blanks, tabs, and newlines are ignored and serve to separate tokens. They have no other semantic meaning. Comments are also ignored.

\subsubsection{Comments}

A single \term{\#} indicates that all succeeding characters shall be considered part of a comment and ignored until a newline is encountered. \\
\\
Immediately following a newline, a series of three \term{\#\#\#} indicates that all succeeding characters shall be considered part of a comment until another series of three \term{\#\#\#} is encountered. Note that newlines are ignored following the \term{\#\#\#}, which essentially delimits multi-line comments.

\subsubsection{Identifiers}

An identifier is a series of alphabetical letters and digits; the first character must be alphabetic. 

\subsubsection{Keywords}

The following identifiers are reserved as keywords/special function and may not be used otherwise:
\begin{table}[htdp]
\center
\begin{tabular}{c c c}
\term{fun} & \term{if} &\term{match} \\
\term{int} & \term{then} & \term{with} \\
\term{bool} & \term{else} &\term{case} \\
\term{float} & \term{in} & \term{string} \\
\term{true} & \term{false} & \term{print} \\
 \term{rand} & \term{and} & \term{or} \\
 \term {given} & \term{print\_line} & \term{seed}
\end{tabular}
\label{default}
\end{table}

The keyword \term{string}  is not currently used, but is reserved for future use.

\subsubsection{Constants}

The reserved boolean constants are \term{true} and \term{false}. The empty list constant is \term{[]} (for all types). 

\subsubsection{Integer Literals}

An integer literal is a sequence of one or more digits, optionally preceded by a minus sign.

Examples of integer literals are \texttt{1337} and \texttt{-42}. 

\subsubsection{Floating-point Literals}

Floating-point decimals consist in an integer part, a decimal part and an exponent part. The integer part is a sequence of one or more digits, optionally preceded by a minus sign. The decimal part is a decimal point followed by zero, one or more digits. The exponent part is the character \texttt e or \texttt E followed by an optional \texttt + or \texttt - sign, followed by one or more digits. The decimal part or the exponent part can be omitted, but not both to avoid ambiguity with integer literals. 

Examples of floating-point constants are \texttt{9000.1}, \texttt{2e-5}, and \texttt{1.4e9}.


\subsection{Types}

The following are the basis data types in YAPPL:\\
\begin{tabular}{l l}
\term{int} & an integer.\\
\term{float} & double-precision floating point.\\
\term{bool} & a boolean value (either \texttt{true} or \texttt{false}).\\
\term{fun} & a function.\\
\end{tabular}\\\\
In addition there are derived array types denoted

\quad \nterm{type} \term{[ ]}


\subsubsection{Non-function Type Declarations}
All bindings must either be declared within a function declaration or declared when bound. A non-function declaration specifies a type and an identifier in the format \nterm{type} \term{:} \nterm{identifier}. Spaces around the colon are optional.  Examples of non-function type declarations:

\begin{alltt}
\quad int:temp
\quad float[]:data
\quad bool : flag
\end{alltt}

\subsubsection{Function Type Declarations}

Function declarations consists of \term{fun} followed by a type declaration for the function in the format \nterm{type} \term{:} \nterm{identifier}. This is followed by zero or more type declarations for arguments of function, which are separated by whitespace. 

Below are examples of function type declarations: 

\begin{alltt}
\quad fun int:add int:a int:b
\quad fun bool:contains float:a float[]:list
\end{alltt}

\subsection{Operations}

\subsubsection{Value binding}

Values are bound to names through the construct 

\begin{alltt}
\quad \nterm{value-decl}\textsuperscript{1} \term{=} \nterm{expr}\textsuperscript{1}  \term{and} \dots \term{and} \nterm{value-decl}\textsuperscript{n} \term{=} \nterm{expr}\textsuperscript{n} \term{in} \nterm{expr}
\end{alltt}

which evaluates  \nterm{expr}\textsuperscript{1} \dots  \nterm{expr}\textsuperscript{n} in the order of declaration and binds the values of those expressions to the names specified in \nterm{value-decl}\textsuperscript{1} \dots \nterm{value-decl}\textsuperscript{n}. The scope of a value declaration begins directly to the right of the declaration.

\subsubsection{Function binding}

The syntax for function binding is identical to that for value binding, except the \nterm{value-decl} is replaced by  a \nterm{function-decl} and any number of \term{=} symbols may be replaced by \term{:=} symbols. The \term{:=} symbol defines a special memoization function. A memoized function is only evaluated once for a set of input values. Once a function is evaluated on those values, it will always return the same value without being reevaluated. 

A literal can only be bound to a function or value once within a program.

\subsubsection{Function evaluation}
Functions are evaluated with the following construct:

\begin{alltt}
\quad \term{\midtilde} \nterm{identifier} [ \nterm{expr}\textsuperscript{1} \dots \nterm{expr}\textsuperscript{n} ] [ \term{given} \nterm{expr} ] 
\end{alltt}

where  \nterm{expr}\textsuperscript{1} \dots \nterm{expr}\textsuperscript{n} are arguments passed to the function. The arguments must match the number and type of the arguments in the declared function being called. The \term{given} \nterm{expr} portion specifies an optional condition that the return value of the function must fulfill. The return value of the function may be referenced within the conditional statement by the special variable \term{\$}. Below is a sample function evaluation that utilizes this construct. The random function \texttt{geom}, which generates a draw from a geometric random variable, takes a single argument between 0 and 1.

\begin{alltt}
\quad \midtilde geom q given \$ > 5
\end{alltt}

Functions are evaluated when they are called. 

\paragraph{A word of caution about function evaluation}

As with function declaration, the arguments passed in are separated by whitespace only.

We make a special note here to remind the reader that whitespace is ignored and to be mindful when calling functions with multiple arguments. As an example, if \texttt{add} expects two integer arguments, when dealing with negative numbers, one must call \texttt{\midtilde add (-1) (-1)}. Neglecting to do so in this situation would have resulted with YAPPL interpreting the second unary minus as a binary minus operator and reducing \texttt{\midtilde add -1 -1} to \texttt{\midtilde add -2}, resulting in an error.

\subsubsection{List construction}

Lists can be constructed using the syntax
\begin{alltt}
\quad \term{[} \nterm{expr}\textsuperscript{1} \term{,} \dots \term{,} \nterm{expr}\textsuperscript{n}  \term{]}
\end{alltt}
Each expression must have the same type. 

\subsubsection{Patterns} 
Patterns are templates that allow selecting values of a given shape and binding identifier names to values. Patterns are used in pattern matching. 

\paragraph{Variable Patterns}
A variable pattern consists of a value identifier. The pattern will match any value, and the value will be bound to the identifier. The pattern \term{\_} will also match any value, but will not result in a binding. A value identifier can only appear once in a pattern.

\paragraph{Constant Patterns}
A pattern consisting of a constant matches the values equal to that constant.

\paragraph{Variant Patterns}
The pattern \nterm{pattern} \term{::} \nterm{pattern} matches non-empty lists whose heads match the first pattern and whose tails match the second pattern. The \term{::} operator for patterns is left-associative.

\subsection{Expressions}

The precedence of expression operators is the same order as they are presented below. Operators in the same grouping (multiplicative, additive, relational etc.) are given the same precedence. Expressions on either side of binary operations must have the same type. 

\subsubsection{Primary expressions}

\paragraph{\nterm{identifier}}
An identifier is a primary expression, provided it has been suitably bound. Its type is specified when bound. 

\paragraph{\nterm{constant}}
A decimal or floating constant is a primary expression. Its type is \texttt{int} in the first case, \texttt{float} in the last. 

\paragraph{\nterm{identifier} \term{[} \nterm{expr} \term{]} }
An identifier followed by an expression in square brackets is a primary expression that yields the value at the \nterm{expr} index of a list, where the expression evaluate to an integer between 0 and one less than the length of the named list. The behavior is unspecified if the index is outside of that range. 

\paragraph{\term{(} \nterm{expr} \term{)}}
A parenthesized expression is a primary expression whose type and value are identical to those of the unadorned expression. 

\subsubsection{Unary operators}

The boolean operator \texttt{!} (negation) and numerical operator \texttt{-} (minus) are unary and group right-to-left.

\begin{alltt}
\quad \term{-} \nterm{expr}
\quad \term{!} \nterm{expr}
\end{alltt}

\subsubsection{Exponential operator}

The exponential operator \texttt{**} is a binary operator that groups right-to-left.
\begin{alltt}
\quad \nterm{expr} \term{**} \nterm{expr}
\end{alltt}
Both expressions must be of type \texttt{float}. The operator evaluates to a \texttt{float}. 

\subsubsection{Multiplicative operators}
The multiplicative operators \texttt{*} (multiplication), \texttt{/} (division), and \texttt{\%} (modulus) are binary and group left-to-right. The  binary \% operator results in the remainder from the division of the first expression by the second. For multiplication and dilation, both operands must be of type \texttt{int} or \texttt{float} and the result is of the same type. Modulus takes integers and returns an integer. The remainder has the same sign as the dividend.

\begin{alltt}
\quad \nterm{expr} \term{*} \nterm{expr}
\quad \nterm{expr} \term{/} \nterm{expr}
\quad \nterm{expr} \term{\%} \nterm{expr}
\end{alltt}

\subsubsection{Additive operators}
The additive operators \texttt{+} (sum) and \texttt{-} (difference) are binary and group left-to-right.
\begin{alltt}
\quad \nterm{expr} \term{+} \nterm{expr}
\quad \nterm{expr} \term{-} \nterm{expr}
\end{alltt}

\subsubsection{Relational operators}
The  relational operators \texttt< (less than), \texttt> (greater than), \texttt{<=} (less than or equal to) and \texttt{>=} (greater than or equal to) all yield false if the specified relation is false, and true if it is true.
\begin{alltt}
\quad \nterm{expr} \term{<} \nterm{expr}
\quad \nterm{expr} \term{>} \nterm{expr}
\quad \nterm{expr} \term{<=} \nterm{expr}
\quad \nterm{expr} \term{>=} \nterm{expr}
\end{alltt}

\subsubsection{Equality operators}
The \texttt{=} (equal to) and the \texttt{!=} (not equal to) operators function as the relational operators above, but have a lower precedence. Therefore, ``\texttt{a < b = c < d}'' is true when \texttt{a < b} and \texttt{c < d} have the same truth value.
\begin{alltt}
\quad \nterm{expr} \term{=} \nterm{expr}
\quad \nterm{expr} \term{!=} \nterm{expr}
\end{alltt}

\subsubsection{Boolean binary operators}
The boolean operators \texttt{\&\&} (conjunction) and \texttt{||} (disjunction) are binary and group left-to-right, with the latter having higher precedence. 
The second operand of \texttt{or} is not evaluated if the value of the first is false.
\begin{alltt}
\quad \nterm{expr} \term{\&\&} \nterm{expr}
\quad \nterm{expr} \term{||} \nterm{expr}
\end{alltt}

\subsubsection{Concatenation operator} 
The concatenation operator yields an list that is the concatenation of the left list at the head of the right list. Both sides must be lists of matching type (e.g.~\term{int[]} or \term{bool[]}). 
\begin{alltt}
\quad \nterm{expr} \term{@} \nterm{expr}
\end{alltt}

\subsubsection{List building operator}
The building operation
\begin{alltt}
\quad \nterm{expr}\textsuperscript{1} \term{::} \nterm{expr}\textsuperscript{2}
\end{alltt}
 yields a list with \nterm{expr}\textsuperscript{1} as the head and \nterm{expr}\textsuperscript{2} as the tail. Thus, if \nterm{expr}\textsuperscript{1} is of type \nterm{type}, then \nterm{expr}\textsuperscript{2} must have type \nterm{type}\term{[]}. The list building operator \term{::} is left-associative. 

\subsubsection{Conditional expression}
The conditional expression evaluates to the second expression if the first is true, otherwise it evaluates to the third expression. The else binds to the closest if.
\begin{alltt}
\quad \term{if} \nterm{expr} \term{then} \nterm{expr} \term{else} \nterm{expr}
\end{alltt}

\subsubsection{Pattern match expression}
The case expression notation yields the expression paired with the first pattern matching the expression to be matched. Each \nterm{pattern}\textsuperscript{1}\dots \nterm{pattern}\textsuperscript{n} should be of the same type.

\begin{alltt}
\quad \term{match} \nterm{expr} \term{with} \nterm{pattern}\textsuperscript{1} \term{->} \nterm{expr}\textsuperscript{1} \term{|} \dots \term{|} \nterm{pattern}\textsuperscript{n} \term{->} \nterm{expr}\textsuperscript{n}
\end{alltt}

\subsubsection{Expression sequencing}
A pair of expressions separated by a semicolon is evaluated left-to-right and the value of the left expression is discarded. The type and value of the result are the type and value of the right operand. This operator groups left to right.
\begin{alltt}
\quad \nterm{expr} \term{;} \nterm{expr}
\end{alltt}

\subsection{Built-in Functions}
There are four built-in functions in YAPPL: \term{rand}, \term{seed}, \term{print}, and \term{print\_line}. These are  reserved keywords. They are called by using the tilde operator just like any other function.

\subsubsection{Random values}
The function \term{rand} takes no arguments and returns a random or pseudo-random number between 0 and 1. \term{seed} also takes no arguments. If it is called before \term{rand}, all subsequent calls of \term{rand} will be seeded with a value based on the current system time. Otherwise all calls to \term{rand} use the same default seed.

\subsubsection{Printing}
Since YAPPL does not currently support the \term{string} type or string literals, or allow for side-effects (excepting \term{rand} and \term{seed}), printing must be achieved explicitly within the language. The \term{print} function takes a single expression of one of the three basic types (\term{int}, \term{bool}, and {float}) or a list of one of those three types as an argument and prints a string representation of that argument to standard output. \term{print\_line} functions like \term{print} but appends a newline.

%that is a list of integers. For each number encountered in the list, \term{print} interprets this as the decimal representation of an ASCII character and prints to standard out the corresponding letter character in the ASCII table. It always returns \term{true}. \\
%\begin{verbatim}
%# prints "Hello World\n" to stdout
%print [72, 101, 108, 108, 111, 32, 87, 111, 114, 100, 10]
%\end{verbatim}

\newpage

\subsection{Grammar}

A summary of the grammar for YAPPL. 

\begin{alltt}
\nterm{expr} = 
\quad \nterm{constant}
\quad \nterm{identifier}
\quad \term{(} \nterm{expr} \term{)}
\quad \nterm{expr \term{;} expr}
\quad \nterm{expr \term{::} expr}
\quad \term{\midtilde} \nterm{identifier} \{ \nterm{expr} \} [ \term{given} \nterm{expr} ]
\quad \nterm{prefix-op expr}
\quad \nterm{expr infix-op expr}
\quad \term{[} \nterm{expr} \{ \term{,} \nterm{expr} \} \term{]}
\quad \nterm{identifier} \term{[} \nterm{expr} \term{]}
\quad \term{if} \nterm{expr} \term{then} \nterm{expr} \term{else} \nterm{expr}
\quad \term{match} \nterm{expr} \term{with} \nterm{pattern-matching}
\quad \term{let} \nterm{value-binding} \{ \term{and} \nterm{value-binding} \} \term{in} \nterm{expr}
\quad \term{fun} \nterm{function-binding} \{ \term{and} \nterm{function-binding} \} \term{in} \nterm{expr}
\quad \term{rand} | \term{seed}
\quad ( \term{print} | \term{print\_line} ) \nterm{expr}

\nterm{type-decl} = 
\quad \nterm{type} \term{:} \nterm{identifier}

\nterm{value-binding} =
\quad  \nterm{value-decl} \term{=} \nterm{expr} 

\nterm{value-decl} \term{=} 
\quad \nterm \nterm{type-decl}

\nterm{function-binding} =
\quad \nterm{function-decl} \nterm{assignment-op} \nterm{expr} 

\nterm{function-decl} =
\quad \nterm{type-decl} \{ \nterm{fun-args-paren} \}

\nterm{fun-args-paren} =
\quad  \nterm{fun-args}
\quad \term{(} \nterm{fun-args} \term{)}

\nterm{fun-args} =
\quad \nterm{fun-args} \nterm{type-decl}  

\nterm{type} =
\quad \nterm{type} \term{[ ]}
\quad \nterm{base-type}
\quad \term{fun} \nterm{type} \{ \nterm{type} \}

\nterm{pattern-matching} =  
\quad [\term{|}]  \nterm{pattern} \term{->} \nterm{expr}  \{ \term{|} \nterm{pattern} \term{->} \nterm{expr}  \}

\nterm{pattern} = 
\quad \term{\_}
\quad \nterm{identifier}
\quad \nterm{constant-of-base-type }
\quad \term{(} \nterm{pattern} \term{)}
\quad \nterm{pattern} \term{::} \nterm{pattern}

\end{alltt}


%identifier:
%\quad identifier-chars \term{but not a reserved word}
%
%identifier-chars:
%\quad letter
%\quad identifier-chars alpha-num
%
%constant: 
%\quad literal
%\quad \term{[]}
%
%literal:
%\quad integer-literal
%\quad float-literal
%\quad bool-literal
%
%int-literal:
%
%float-literal:
%
%bool-literal: one of
%\quad \term{true false}
%
%type:
%\quad basic-type
%\quad array-type
%
%array-type:
%\quad basic-type\term{[]}
%
%basic-type: one of
%\quad \term{int float bool}

%\nterm{assignment-op} =
%\quad \term{= }
%\quad \term{:=  }
%letter: one of
%\quad \term{a ... z A ... Z}
%
%digit: one of
%\quad \term{0 ... 9}
%
%alpha-num:
%\quad letter
%\quad digit

%\nterm{prefix-op} =
%\quad \term{!}
%\quad \term{-}
%
%\nterm{infix-op} =
%\quad \term{and}
%\quad \term{or}
%\quad \term{+}
%\quad \term{-}
%\quad \term{*}
%\quad \term{/}
%\quad \term{\%}
%\quad \term{@}
%\quad \term{=}
%\quad \term{!=}
%\quad \term{<}
%\quad \term{>}
%\quad \term{>=}
%\quad \term{<=}


%\nterm{assignment-op} =
%\quad \term{=}
%\quad \term{:=}

