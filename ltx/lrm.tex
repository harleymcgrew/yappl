Outstanding issues:
\begin{itemize}
\item pattern matching (Harley)
\item consistent formatting (Hans)
\item consistency between document and grammar (David)
\item built-in functions -- print and rand (David)
\item final check (Jonathan)
\end{itemize}

\subsection{Notation}

Through the document, \nterm{nonterminals} are in brown italics and \term{terminals} are in light blue monospace. Regular expression-like constructs are used to simplify grammar presentation and are in black. Brackets \texttt{[]} are used to indicate optional parts of productions, curly braces \texttt{\{\}} indicate portions of productions that can appear zero or more times, and parentheses \texttt{()} indicate grouping. 

\subsection{Lexical conventions}

As our syntax is inspired by OCaml, many of our lexical conventions follow those of that language. YAPPL has four kinds of tokens: identifiers, keywords, constants, and expression operators. Whitespace such as blanks, tabs, and newlines are ignored and serve to separate tokens. Comments are also ignored.

\subsubsection{Comments}

A single \term{\#} indicates that all succeeding characters shall be considered part of a comment and ignored until a newline is encountered. \\
\\
Immediately following a newline, a series of three \term{\#\#\#} indicates that all succeeding characters shall be considered part of a comment until another series of three \term{\#\#\#} is encountered. Note that newlines are ignored following the \term{\#\#\#}, which essentially delimits multi-line comments.

\subsubsection{Identifiers}

An identifier is a series of alphabetical letters and digits; the first character must be alphabetic. 

\subsubsection{Keywords}

The following identifiers are reserved as keywords and may not be used otherwise:
\begin{table}[htdp]
\center
\begin{tabular}{c c c}
\term{fun} & \term{if} &\term{match} \\
\term{int} & \term{then} & \term{with} \\
\term{bool} & \term{else} &\term{case} \\
\term{float} & \term{in} & \term{string} \\
\term{true} & \term{false} & \term{print} \\
 \term{rand} & \term{and} & \term{or} 
\end{tabular}
\label{default}
\end{table}

The keyword \term{string}  is not currently used, but is reserved for future use.

\subsubsection{Constants}

The reserved boolean constants are \term{true} and \term{false}. The empty list constant is \term{[]} (for all types). 

\subsubsection{Integer Literals}

An integer literal is a sequence of one or more digits, optionally preceded by a minus sign.

Examples of integer literals are \texttt{1337} and \texttt{-42}. 

\subsubsection{Floating-point Literals}

Floating-point decimals consist in an integer part, a decimal part and an exponent part. The integer part is a sequence of one or more digits, optionally preceded by a minus sign. The decimal part is a decimal point followed by zero, one or more digits. The exponent part is the character e or E followed by an optional + or - sign, followed by one or more digits. The decimal part or the exponent part can be omitted, but not both to avoid ambiguity with integer literals. 

Examples of floating-point constants are \texttt{9000.1}, \texttt{2e-5}, and \texttt{1.4e9}.


\subsection{Types}

The following are the basis data types in YAPPL:\\
\begin{tabular}{l l}
\term{int} & an integer.\\
\term{float} & double-precision floating point.\\
\term{bool} & a boolean value (either \texttt{true} or \texttt{false}).\\
\term{fun} & a function.\\
\end{tabular}\\\\
In addition there are derived array types denoted

\quad \nterm{type} \term{[ ]}

YAPPL does not support conversion between types. All types are immutable.

\subsubsection{Non-function Type Declarations}
All bindings must either be declared within a function declaration or declared when bound. A non-function declaration specifies a type and an identifier in the format \nterm{type} \term{:} \nterm{identifier}. Spaces around the colon are optional.  Examples of non-function type declarations:

\texttt{int:temp}\\
\texttt{float[]:data}\\
\texttt{bool : flag}

\subsubsection{Function Type Declarations}

Function declarations consists of \term{fun} followed by a type declaration for the return type, followed by zero or more type declarations for arguments of function. Optionally, parentheses may surround the type declarations. Examples of function type declarations: 

\texttt{fun int:add int:a int:b}\\
\texttt{fun bool:contains (float:a float[]:list)}

\subsection{Operations}

\subsubsection{Value binding}

Values are bound to names through the construct 

\begin{alltt}
\quad \nterm{value-decl}\textsuperscript{1} \term{=} \nterm{expr}\textsuperscript{1}  \term{and} \dots \term{and} \nterm{value-decl}\textsuperscript{n} \term{=} \nterm{expr}\textsuperscript{n}  \term{in} \nterm{expr}
\end{alltt}

which evaluates  \nterm{expr}\textsuperscript{1} \dots  \nterm{expr}\textsuperscript{n} in an unspecified order and binds the values of those expressions to the names specified in \nterm{value-decl}\textsuperscript{1} \dots \nterm{value-decl}\textsuperscript{n}.

\subsubsection{Function binding}

The syntax for function binding is identical to that for value binding, except \nterm{value-decl} is replaced by \nterm{function-decl} and any number of \term{=} symbols may be replaced by \term{:=} symbols. The \term{:=} symbol defines a special memoization function. A memoized function is only evaluated once for a set of input values. Once function is evaluated on those values, it will always return the same value without being reevaluated. 

\subsubsection{Function evaluation}
Functions are evaluated with the following construct:

\begin{alltt}
\quad \term{\midtilde} \nterm{identifier} [ \nterm{expr}\textsuperscript{1} \dots \nterm{expr}\textsuperscript{n} [ \term{|} \nterm{expr} ] ]
\end{alltt}

where  \nterm{expr}\textsuperscript{1} \dots \nterm{expr}\textsuperscript{n}  are optional arguments passed to the function and \term{|} \nterm{expr} specifies an optional condition that the return value of the function must fulfill. The return value of the function may be referenced within the condition by the special variable \term{\$}.

\subsubsection{Patterns} 
Patterns are templates that allow selecting values of a given shape and binding identifier names to values. Patterns are used in pattern matching. 

\paragraph{Variable Patterns}
A variable pattern consists of a value identifier. The pattern will match any value, and the value will be bound to the identifier. The pattern \term{\_} will also match any value, but will not result in a binding. A value identifier can only appear once in a pattern.

\paragraph{Constant Patterns}
A pattern consisting of a constant matches the values equal to that constant.

\paragraph{Variant Patterns}
The pattern \nterm{pattern} \term{::} \nterm{pattern} matches non-empty lists whose heads match the first pattern and whose tails match the second pattern. The \term{::} operator is right associative.

\subsection{Expressions}

The precedence of expression operators is the same order as they are presented below. Operators in the same grouping (multiplicative, additive, relational etc.) are given the same precedence. Expressions on either side of binary operations must have the same type. 

\subsubsection{Primary expressions}
Primary expressions such as \term{\midtilde} involving function calls group left to right.

\paragraph{\nterm{identifier}}
An identifier is a primary expression, provided it has been suitably bound. Its type is specified when bound. 

\paragraph{\nterm{constant}}
A decimal or floating constant is a primary expression. Its type is \texttt{int} in the first case, \texttt{float} in the last. 

\paragraph{\nterm{identifier}\term{[}\nterm{expr}\term{]}}
An identifier followed by an expression in square brackets is a primary expression that yields the value at the \texttt{int}  index of a list.

\paragraph{\term{(} \nterm{expr} \term{)}}
A parenthesized expression is a primary expression whose type and value are identical to those of the unadorned expression. 

\subsubsection{Multiplicative operators}
The multiplicative operators \texttt{*} (multiplication), \texttt{/} (division), and \texttt{\%} (modulus) are binary and group left-to-right. The  binary \% operator results in the remainder from the division of the first expression by the second. Both operands must be type \texttt{int} and the result is \texttt{int}. The remainder has the same sign as the dividend.

\begin{alltt}
\quad \nterm{expr} \term{*} \nterm{expr}
\quad \nterm{expr} \term{/} \nterm{expr}
\quad \nterm{expr} \term{\%} \nterm{expr}
\end{alltt}

\subsubsection{Additive operators}
The additive operators \texttt{+} (sum) and \texttt{-} (difference) are binary and group left-to-right.
\begin{alltt}
\quad \nterm{expr} \term{+} \nterm{expr}
\quad \nterm{expr} \term{-} \nterm{expr}
\end{alltt}

\subsubsection{Relational operators}
The  relational operators < (less than), > (greater than), <= (less than or equal to) and >= (greater than or equal to) all yield false if the specified relation is false, and true if it is true. These operators group left-to-right. However, this is not particularly meaningful. For example, in a<b<c, a<b will evaluate to either true or false, and then the resulting expression is type mismatched or meaningless.

\begin{alltt}
\quad \nterm{expr} \term{<} \nterm{expr}
\quad \nterm{expr} \term{>} \nterm{expr}
\quad \nterm{expr} \term{<=} \nterm{expr}
\quad \nterm{expr} \term{>=} \nterm{expr}
\end{alltt}

\subsubsection{Equality operators}
The \texttt{=} (equal to) and the \texttt{!=} (not equal to) operators function as the relational operators above, but have a lower precedence. Therefore, "a<b = c<d" is true when a<b and c<d have the same truth value.
\begin{alltt}
\quad \nterm{expr} \term{=} \nterm{expr}
\quad \nterm{expr} \term{!=} \nterm{expr}
\end{alltt}

\subsubsection{Boolean operators}
The boolean operators \texttt{and} (conjunction) and \texttt{or} (disjunction) are binary and group left-to-right. The boolean operator \texttt{!} (negation) is unary and groups right-to-left.
The second operand or \texttt{or} may not be evaluated if the value of the first is false.
\begin{alltt}
\quad \nterm{expr} \term{and} \nterm{expr}
\quad \nterm{expr} \term{or} \nterm{expr}
\quad \term{!} \nterm{expr}
\end{alltt}

\subsubsection{Concatenation operator} 
The concatenation operator yields an list that is the concatenation of the left list at the head of the right list. Both sides must be lists of matching type (i.e. \term{fun[]}, \term{int[]}, \term{bool[]}, \term{float[]}.)
\begin{alltt}
\quad \nterm{expr} \term{@} \nterm{expr}
\end{alltt}

\subsubsection{Conditional expression}
The conditional expression evaluates to the second expression if the first is true, otherwise it evaluates to the third expression. If nesting another conditional, parentheses are strongly recommended to prevent ambiguity.
\begin{alltt}
\quad \term{if} \nterm{expr} \term{then} \nterm{expr} [\term{else} \nterm{expr}]
\end{alltt}

\subsubsection{Pattern match expression}
The case expression notation yields the expression paired with the first pattern matching the expression to be matched.

\begin{alltt}
\quad \term{match} \nterm{expr} \term{with} \nterm{pattern} \term{->} \nterm{expr} \term{|} \dots \term{|} \nterm{pattern} \term{->} \nterm{expr}
\end{alltt}

\subsubsection{Expression sequencing}
A pair of expressions separated by a semicolon is evaluated left-to-right and the value of the left expression is discarded. The type and value of the result are the type and value of the right operand. This operator groups left to right.
\begin{alltt}
\quad \nterm{expr} \term{;} \nterm{expr}
\end{alltt}

\subsection{Built-in Functions}
There are two built-in functions in YAPPL: \term{rand} and \term{print}. These are both reserved keywords. 
\subsubsection{rand}
The function \term{rand} takes no arguments and returns a random or pseudo-random number between 0 and 1. 

\subsubsection{print}
Since YAPPL does not use the \term{string} type nor include string literals, printing must be achieved through other means. The \term{print} function takes a single argument that is a list of integers. For each number encountered in the list, \term{print} interprets this as the decimal representation of an ASCII character and prints to standard out the corresponding letter character in the ASCII table. It always returns \term{true}. \\
\begin{verbatim}
# prints "Hello World\n" to stdout
print [72, 101, 108, 108, 111, 32, 87, 111, 114, 100, 10]
\end{verbatim}

\newpage

\subsection{Grammar}

%identifier:
%\quad identifier-chars \term{but not a reserved word}
%
%identifier-chars:
%\quad letter
%\quad identifier-chars alpha-num
%
%constant: 
%\quad literal
%\quad \term{[]}
%
%literal:
%\quad integer-literal
%\quad float-literal
%\quad bool-literal
%
%int-literal:
%
%float-literal:
%
%bool-literal: one of
%\quad \term{true false}
%
%type:
%\quad basic-type
%\quad array-type
%
%array-type:
%\quad basic-type\term{[]}
%
%basic-type: one of
%\quad \term{int float bool}

\begin{alltt}
\nterm{expr} = 
\quad \nterm{constant}
\quad \nterm{identifier}
\quad \term{(} \nterm{expr} \term{)}
\quad \nterm{expr \term{;} expr}
\quad \nterm{expr \term{::} expr}
\quad \term{\midtilde} \nterm{identifier} \{ \nterm{expr} \} [ \term{|} \nterm{expr} ]
\quad \nterm{prefix-op expr}
\quad \nterm{expr infix-op expr}
\quad \term{[} \nterm{expr} \{ \term{,} \nterm{expr} \} \term{]}
\quad \term{if} \nterm{expr \term{then} expr} [ \term{else} \nterm{expression} ]
\quad \term{match} \nterm{expression} \term{with} \nterm{pattern-matching}
\quad \nterm{value-binding} \{ \term{and} \nterm{value-binding} \} \term{in} \nterm{expr}
\quad \nterm{function-binding} \{ \term{and} \nterm{function-binding} \} \term{in} \nterm{expr}

\nterm{value-binding} =
\quad \nterm{value-decl} \term{=} \nterm{expr} 

\nterm{function-binding} =
\quad \nterm{function-decl} \nterm{assignment-op} \nterm{expr} 

\nterm{type-decl} = 
\quad \nterm{var-decl}
\quad \nterm{function-decl}

\nterm{function-decl} =
\quad \term{fun} \nterm{type-decl}  \{ \nterm{type-decl} \}

\nterm{var-decl} = 
\quad \nterm{type} \term{:} \nterm{identifier}

\nterm{type} =
\quad \nterm{type} \term{[ ]}
\quad \nterm{base-type}
\quad \term{fun} \nterm{type} \{ \nterm{type} \}

\nterm{pattern-matching} =  
\quad [\term{|}]  \nterm{pattern} \term{->} \nterm{expression}  \{ \term{|} \nterm{pattern} \term{->} \nterm{expression}  \}

\nterm{pattern} = 
\quad \term{\_}
\quad \nterm{identifier}
\quad \nterm{constant}
\quad \term{(} \nterm{pattern} \term{)}
\quad \nterm{pattern} \term{::} \nterm{pattern}

\end{alltt}

%\nterm{assignment-op} =
%\quad \term{= }
%\quad \term{:=  }
%letter: one of
%\quad \term{a ... z A ... Z}
%
%digit: one of
%\quad \term{0 ... 9}
%
%alpha-num:
%\quad letter
%\quad digit

%\nterm{prefix-op} =
%\quad \term{!}
%\quad \term{-}
%
%\nterm{infix-op} =
%\quad \term{and}
%\quad \term{or}
%\quad \term{+}
%\quad \term{-}
%\quad \term{*}
%\quad \term{/}
%\quad \term{\%}
%\quad \term{@}
%\quad \term{=}
%\quad \term{!=}
%\quad \term{<}
%\quad \term{>}
%\quad \term{>=}
%\quad \term{<=}


%\nterm{assignment-op} =
%\quad \term{=}
%\quad \term{:=}

