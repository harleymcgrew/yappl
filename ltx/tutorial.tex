YAPPL was designed to make it easy to work with probabilities. The following tutorial will give a quick tour of the basics of the language and eventually build up to define a probability distribution and draw samples from it. At the end we highlight constructs unique to YAPPL. 

\subsection{Basic values}

Let's begin by exploring the simple concepts of the language. There are three basic data types: integers, floating-point numbers, and booleans. Here's how to declare values as each one of these:
\begin{floatbox}
\begin{verbatim}
int:num
float:number_with_a_dot
bool:truthful
\end{verbatim}
\caption{Value declarations.}
\end{floatbox}

It would be helpful if we could give these names values and actually use them. To do so, we need to discuss scope. When a value is bound to a name, you must also specify where it is defined. This is done explicitly through the \texttt{let\dots in} construction:

\begin{floatbox}
\begin{verbatim}
let int:num = 5 in
  num + 5
\end{verbatim}
\caption{Value definition and scoping.}
\end{floatbox}

The scope of \texttt{num} is the block after \texttt{in}.

\subsection{Functions}

Function declaration and definition looks similar to its value counterparts. Scoping works in a similar way, though a function's scope begins immediately after the \texttt{=}, so it may be called recursively.  There is no need to explicitly define a function as being recursive. 
\begin{floatbox}
\begin{alltt}
fun int:add int:a int:b = 
   a + b
in 
  {\midtilde}print\_line {\midtilde}add 1 2
\end{alltt}
\caption{A simple function.}
\end{floatbox}

This creates a function \texttt{add} that returns an integer and takes in two integers as arguments. The body of the function is after the \texttt{=} sign. A function returns whatever value its body expression evaluates to. 

Functions are called using the function evaluation operator, tilde (\midtilde). 

\subsubsection{Built-in functions}

The above code also demonstrates a built-in function of YAPPL, \texttt{print\_line}, which takes in a single argument to print. The other built-in functions are \texttt{rand} and \texttt{seed}, which we will use later in this tutorial.

\subsection{Expressions}

Function and value declarations and definitions are expressions, and so are function evaluations. Sequences of expressions are separated by semicolons. The \texttt{if} statement will be useful in this tutorial:
\begin{floatbox}
\begin{alltt}
if x < 10 then
  {\midtilde}print\_line x
else
  {\midtilde}print\_line 10
\end{alltt}
\caption{An \texttt{if} statement.}
\end{floatbox}

For a more formal description, please see the Language Reference Manual. 

\subsection{A simple distribution}

We now know enough to construct a function that takes a sample from a distribution. We will use the geometric distribution because it is a simple enough construct to use in an exemplary program, but still interesting. To remind the reader, a sample from a geometric distribution will return the number of Bernoulli trials needed to obtain a success, given some probability of success $p$. 

Our approach will generate a random number for a number of iterations. Each time a random number is generated, we will increase our integer return value. The function will continue to iterate until the random number is within $[0, p)$. We return the final value. We implement the increment recursively:

\begin{floatbox}
\begin{alltt}
fun int:geom float:p int:i =
    if {\midtilde}rand < p then
      i
    else
      {\midtilde}geom p (i+1)
\end{alltt}
\caption{The basics of sampling from a geometric distribution.}
\end{floatbox}

This is a good start, except we now mention a nuance of YAPPL. For \texttt{rand} to return unique numbers on each execution of a program, \texttt{seed} must first be called. Furthermore, the interface is not very clean: it would be better to have a function that takes a single parameter that is some probability $p$. Finally, we would actually like to call this and run the program. The final code for \texttt{geom.ypl} is below.

\begin{floatbox}
\begin{alltt}
 {\midtilde}seed;
 fun int:geom float:p =
   fun int:geom_helper float:orig_p int:i =
     if {\midtilde}rand < orig_p then
       i
     else
       {\midtilde}geom_helper orig_p (i+1)
   in 
     {\midtilde}geom_helper p 1
 in
   {\midtilde}print_line {\midtilde}geom 0.1
\end{alltt}
\caption{The full \texttt{geom.ypl}.}
\end{floatbox}

\subsection{Compilation and execution}

To compile a YAPPL file, pass it into the YAPPL compiler, which produces an executable. The following is the basic breakdown of compiling and executing a \texttt{.ypl} file from the command line:

\begin{floatbox}
\begin{alltt}
\$ ./yapplc geom.ypl
\$ ./geom
\$ 8
\end{alltt}
\caption{Compilation and execution.}
\end{floatbox}

\subsection{Conditionals}

With the basics of the language covered, we now give an overview of special constructs built into the language and how to use them. 

Below we explain conditionals, which can be specified after any function evaluation. The \texttt{given} keyword specifies a condition that must be met. The special symbol \texttt{\$} references the return value of a function within the context of \texttt{given}.

Below is a line of code that returns a random number less than .5 and the last line of \texttt{geom.ypl} after modifying it to say we want \texttt{geom} to return a value great than some other variable \texttt{a}.

\begin{floatbox}
\begin{alltt}
 {\midtilde}rand given \$ < .5
 
 {\midtilde}geom 0.1 given \$ > a
\end{alltt}
\caption{Two conditionals with \texttt{given}.}
\end{floatbox}

\subsection{Memoization \& returning functions}

You can create a memoized function using the \texttt{:=} operator instead of the \texttt{=} operator during function definition. A memoized function will remember the value it returned for previously evaluated parameter values, and it will always return this same value. Below we give a memoized version of our geometric sampler. 

The code below also highlights another useful feature of YAPPL: the ability pass around and return functions. The notation \texttt{(fun int int):geom\_list\_gen} indicates that the return type of \texttt{geom\_list\_gen} is a function that returns an integer value and takes a single integer value as a parameter.

\begin{floatbox}
\caption{A memoized \texttt{geom.ypl}.}
\begin{alltt}
 fun int:geom float:q =
   fun int:geom_helper float:orig_q int:i =
     if {\midtilde}rand < orig_q then
       i
     else
       {\midtilde}geom_helper orig_q (i+1)
   in 
     {\midtilde}geom_helper q 1
 in
 fun (fun int int):geom_list_gen float:p = 
     fun int:geom_list int:n :=
     	{\midtilde}geom p
     in
     geom_list
 in
 {\midtilde}seed;
 let (fun int int):g = {\midtilde}geom_list_gen 0.5 in
 {\midtilde}print_line [{\midtilde}g 1, {\midtilde}g 1,{\midtilde}g 3, {\midtilde}g 3]
 
\end{alltt}

Sample output: \texttt{[4, 4, 2, 2]}
\end{floatbox}

Note that \texttt{geom} is the same as before; however, we define a \texttt{geom\_list\_gen} function that creates a returns a memoized version of  \texttt{geom}. \texttt{g} is an instance of \texttt{geom\_list\_gen}. Although we are taking four different samples, because of memoization, the first two elements of the printed list will always have the same value, as well as the last two elements.
\newpage
\subsection{Pattern matching}

YAPPL also supports pattern matching. The following code uses the standard library function \texttt{fflip}, which returns either \texttt{true} or \texttt{false}. However, instead of printing these boolean values, we use pattern matching to print \texttt{0} or \texttt{1} instead.

\begin{floatbox}
\caption{Basic pattern matching.}
\begin{alltt}
 {\midtilde}seed;
 match {\midtilde}fflip with
     true -> {\midtilde}print_line 1
   | false -> {\midtilde}print_line 0
\end{alltt}
\end{floatbox}

Another example of pattern matching is this function which calculates the sum of a list of integers. 
\begin{floatbox}
\caption{Pattern matching with lists.}
\begin{alltt}
fun int:add int[]:nums =
   match nums with 
   	   n :: rest -> n + {\midtilde}add rest
     | []        -> 0
in
 {\midtilde}print_line {\midtilde}add [1,2,3,4,5]
\end{alltt}
\end{floatbox}


%# single line comment
%### multi
%    line
%	comment
%###
%
%# value definition
%float:q = .9;
%# q is defined as 0.9 in the global scope
%
%# sample binding;
%int:x = ~geom q | @ > 5;
%# x is bound to the return value of the function geom evaluated with
%# parameter q; the �| @ > 5� means that the return value of ~geom q is 
%# conditions being greater than 5
%
%# function definition
%int:oneOrTwo float:q = geom q | @ = 1 or @ = 2;
%# a function that samples from geom q, conditional on the sample being 1 or 2
%
%# function equivalent to above
%fun int:oneOrTwo2 int:q =
%   int:x = ~geom q in
%   if x = 1 or x = 2 then
%x
%   else
%~oneOrTwo2 q;
%
%fun bool gtFive int:x = x > 5;
%
%# this is a memoized function
%fun int:f int:n := ~geom .9 | gtFive @;
%~f 0;
%-> 16
%~f 1;
%-> 6
%# later, ~f 0 still returns 16
%~f 0;
%-> 16 
%
%fun int[]:apply (fun int int):f int[]:a =
%	match(a) with
%case x :: rest -> ~f x :: ~apply f rest
%		case [] 	   -> ();
%# apply has type fun int[] ((fun int int) int[])
%# f has type fun int (int)